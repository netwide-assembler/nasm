%
% vim: ts=4 sw=4 et
%
\xchapter{mixsize}{Mixing 16 and 32 Bit Code}

This chapter tries to cover some of the issues, largely related to
unusual forms of addressing and jump instructions, encountered when
writing operating system code such as protected-mode initialisation
routines, which require code that operates in mixed segment sizes,
such as code in a 16-bit segment trying to modify data in a 32-bit
one, or jumps between different-size segments.

\xsection{mixjump}{Mixed-Size Jumps}
\index{jumps!mixed-size}
\index{operating system, writing}
\index{writing operating systems}

The most common form of \textindex{mixed-size instruction} is the one
used when writing a 32-bit OS: having done your setup in 16-bit mode,
such as loading the kernel, you then have to boot it by switching into
protected mode and jumping to the 32-bit kernel start address. In a
fully 32-bit OS, this tends to be the \emph{only} mixed-size
instruction you need, since everything before it can be done in pure
16-bit code, and everything after it can be pure 32-bit.

This jump must specify a 48-bit far address, since the target
segment is a 32-bit one. However, it must be assembled in a 16-bit
segment, so just coding, for example,

\begin{lstlisting}
jmp 0x1234:0x56789ABC           ; wrong!
\end{lstlisting}

will not work, since the offset part of the address will be
truncated to \code{0x9ABC} and the jump will be an ordinary 16-bit far
one.

The Linux kernel setup code gets round the inability of \code{as86} to
generate the required instruction by coding it manually, using
\code{DB} instructions. NASM can go one better than that, by actually
generating the right instruction itself. Here's how to do it right:

\begin{lstlisting}
jmp     dword 0x1234:0x56789ABC ; right
\end{lstlisting}

\indexcode{JMP DWORD}The \code{DWORD} prefix (strictly speaking,
it should come \emph{after} the colon, since it is declaring the
\emph{offset} field to be a doubleword; but NASM will accept either
form, since both are unambiguous) forces the offset part to be treated
as far, in the assumption that you are deliberately writing a jump from
a 16-bit segment to a 32-bit one.

You can do the reverse operation, jumping from a 32-bit segment to a
16-bit one, by means of the \code{WORD} prefix:

\begin{lstlisting}
jmp     word 0x8765:0x4321  ; 32 to 16 bit
\end{lstlisting}

If the \code{WORD} prefix is specified in 16-bit mode, or the
\code{DWORD} prefix in 32-bit mode, they will be ignored, since each is
explicitly forcing NASM into a mode it was in anyway.

\xsection{mixaddr}{Addressing Between Different-Size Segments}
\index{addressing!mixed-size}
\index{mixed-size addressing}

If your OS is mixed 16 and 32-bit, or if you are writing a DOS
extender, you are likely to have to deal with some 16-bit segments
and some 32-bit ones. At some point, you will probably end up
writing code in a 16-bit segment which has to access data in a
32-bit segment, or vice versa.

If the data you are trying to access in a 32-bit segment lies within
the first 64K of the segment, you may be able to get away with using
an ordinary 16-bit addressing operation for the purpose; but sooner
or later, you will want to do 32-bit addressing from 16-bit mode.

The easiest way to do this is to make sure you use a register for
the address, since any effective address containing a 32-bit
register is forced to be a 32-bit address. So you can do

\begin{lstlisting}
mov     eax,offset_into_32_bit_segment_specified_by_fs
mov     dword [fs:eax],0x11223344
\end{lstlisting}

This is fine, but slightly cumbersome (since it wastes an
instruction and a register) if you already know the precise offset
you are aiming at. The x86 architecture does allow 32-bit effective
addresses to specify nothing but a 4-byte offset, so why shouldn't
NASM be able to generate the best instruction for the purpose?

It can. As in \nref{mixjump}, you need only prefix the address
with the \code{DWORD} keyword, and it will be forced to be a 32-bit
address:

\begin{lstlisting}
mov     dword [fs:dword my_offset],0x11223344
\end{lstlisting}

Also as in \nref{mixjump}, NASM is not fussy about whether the
\code{DWORD} prefix comes before or after the segment override, so
arguably a nicer-looking way to code the above instruction is

\begin{lstlisting}
mov     dword [dword fs:my_offset],0x11223344
\end{lstlisting}

Don't confuse the \code{DWORD} prefix \emph{outside} the square brackets,
which controls the size of the data stored at the address, with the
one \code{inside} the square brackets which controls the length of the
address itself. The two can quite easily be different:

\begin{lstlisting}
mov     word [dword 0x12345678],0x9ABC
\end{lstlisting}

This moves 16 bits of data to an address specified by a 32-bit
offset.

You can also specify \code{WORD} or \code{DWORD} prefixes along with the
\code{FAR} prefix to indirect far jumps or calls. For example:

\begin{lstlisting}
call    dword far [fs:word 0x4321]
\end{lstlisting}

This instruction contains an address specified by a 16-bit offset;
it loads a 48-bit far pointer from that (16-bit segment and 32-bit
offset), and calls that address.

\xsection{mixother}{Other Mixed-Size Instructions}

The other way you might want to access data might be using the
string instructions (\code{LODSx}, \code{STOSx} and so on) or the
\code{XLATB} instruction. These instructions, since they take no
parameters, might seem to have no easy way to make them perform
32-bit addressing when assembled in a 16-bit segment.

This is the purpose of NASM's \codeindex{a16}, \codeindex{a32} and
\codeindex{a64} prefixes. If you are coding \code{LODSB} in a 16-bit
segment but it is supposed to be accessing a string in a 32-bit segment,
you should load the desired address into \code{ESI} and then code

\begin{lstlisting}
a32     lodsb
\end{lstlisting}

The prefix forces the addressing size to 32 bits, meaning that
\code{LODSB} loads from \code{[DS:ESI]} instead of \code{[DS:SI]}.
To access a string in a 16-bit segment when coding in a 32-bit one,
the corresponding \code{a16} prefix can be used.

The \code{a16}, \code{a32} and \code{a64} prefixes can be applied to
any instruction in NASM's instruction table, but most of them can
generate all the useful forms without them. The prefixes are necessary
only for instructions with implicit addressing: \code{CMPSx},
\code{SCASx}, \code{LODSx}, \code{STOSx}, \code{MOVSx}, \code{INSx},
\code{OUTSx}, and \code{XLATB}. Also, the various push and pop
instructions (\code{PUSHA} and \code{POPF} as well as the more usual
\code{PUSH} and \code{POP}) can accept \code{a16}, \code{a32} or
\code{a64} prefixes to force a particular one of \code{SP}, \code{ESP} or
\code{RSP} to be used as a stack pointer, in case the stack segment in
use is a different size from the code segment.

\code{PUSH} and \code{POP}, when applied to segment registers in 32-bit
mode, also have the slightly odd behaviour that they push and pop 4
bytes at a time, of which the top two are ignored and the bottom two
give the value of the segment register being manipulated. To force
the 16-bit behaviour of segment-register push and pop instructions,
you can use the operand-size prefix \codeindex{o16}:

\begin{lstlisting}
o16 push    ss
o16 push    ds
\end{lstlisting}

This code saves a doubleword of stack space by fitting two segment
registers into the space which would normally be consumed by pushing
one.

(You can also use the \codeindex{o32} prefix to force the 32-bit behaviour
when in 16-bit mode, but this seems less useful.)
